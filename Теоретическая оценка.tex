\documentclass{report}

\usepackage[left=1cm,right=1cm,top=1cm,bottom=2cm]{geometry}

\usepackage[utf8]{inputenc}
\usepackage[russian]{babel}
\usepackage{amsmath}
\usepackage{amssymb}
\usepackage{wasysym}
\usepackage{bbold}
\usepackage{sectsty}

\newcommand{\N}{\mathbb{N}}
\newcommand{\R}{\mathbb{R}}
\renewcommand{\phi}{\varphi}

\DeclareMathOperator{\erf}{erf}

\addto\captionsrussian{\renewcommand{\chaptername}{Задача}}
\chapternumberfont{\LARGE}
\chaptertitlefont{\Large}


\title{Лабораторная 2}
\author{Григоренко Павел, M3238}
\date{03-06-2023}


\begin{document}
\maketitle
\chapter{
Случайная величина X принимает значения $-1$, $0$, $1$ с вероятностями $p_1$, $p_2$ и $p_3$ соответственно
Какие условия нужно наложить на $p_1$, $p_2$, $p_3$, чтобы случайная величина $X$
была представима в виде суммы двух независимых одинаково распределенных случайных величин?
}

Пусть эти величины~--- $X_1$ и $X_2$.
Очевидно, что они должны быть невырожденными (иначе их сумма тоже вырожденная).
Также нетрудно понять, что они не могут принимать более двух различных значений,
ведь тогджа их сумма будет принимать не менее трёх различных значений.

Итого, мы выяснили, что $X_1$ и $X_2$~--- дискретные случайные величины,
независимые и с одинаковым распределением (два различных значения).
Думаю, очевидно, что эти значения~--- это $\pm0.5$.

Пусть значение $+0.5$ принимается с вероятность $p$ ($0<p<1$).
Тогда значение $-0.5$ имеет вероятность $1-p$.

Так как $X_1$ и $X_2$ независимы, то
\[
    p_3=P\left( X=1 \right)=P\left(X_1=+0.5 \land X_2=+0.5\right)=
    P\left(X_1=+0.5\right)\cdot P\left(X_2=+0.5\right)=p^2
\]

Аналогично,  $p_1=P\left( X=-1 \right)=\left( 1-p \right)^2$.

Ну и $p_2=P\left( X=0 \right)=2p(1-p)$.

Получается, что на $p_1$ и $p_2$ должны удовлетворять следующим условиям:
\[
\begin{cases}
p_1=\left( 1-\sqrt{p_3} \right)^2 \\
p_2=2\sqrt{p_3}\left( 1-\sqrt{p_3} \right)
\end{cases}
\]

Но так как $p_1+p_2+p_3=1$, то условия выше немного избыточны.
Достаточно просто потребовать $p_1=\left( 1-\sqrt{p_3} \right)^2$.
(Ну и также $0<p_3<1$, чтобы $0\notin\{p_1,p_2,p_3\}$).

Ответ:
\[
\begin{cases}
p_1=\left( 1-\sqrt{p_3} \right)^2 \\
0<p_3<1
\end{cases}
\]



\chapter{
Пусть пара случайных величин $(U, V)$ имеет равномерное распределение на единичном круге и $W = U^2 + V^2$.
Положим
\[
    X=U\sqrt{-\frac{2\log W}{W}},\quad
    Y=V\sqrt{-\frac{2\log W}{W}}
\]
Показать, что $X$, $Y$~--- независимы и являются стандартными гауссовскими величинами.
}

Полярные координаты!
\begin{gather*}
    U=R\cos\phi\\
    V=R\sin\phi\\
    \phi\in\left[0;2\pi\right]\\
    R\in\left[0;1\right]
\end{gather*}
Чтобы сохранить равномерное распределение по единичному кругу, берём следующие плотности вероятностей:
\begin{gather*}
    f_\phi(\psi)=\frac{1}{2\pi}\\
    f_R(r)=2r
\end{gather*}

Подставим это в формулы для $W$, $X$ и $Y$:
\[ W=U^2+V^2=R^2\cos^2\phi+R^2\sin^2\phi=R^2 \]
\[ X=U\sqrt{-\frac{2\log W}{W}}=
R\cos\phi\sqrt{-\frac{2\log R^2}{R^2}}=
\cos\phi\sqrt{-2\log R^2}=
\cos\phi\sqrt{-4\log R}=
2\cos\phi\sqrt{-\log R} \]
\[
 Y=U\sqrt{-\frac{2\log W}{W}}=\dots=2\sin\phi\sqrt{-\log R}
\]

Давайте найдём функцию распределения для $Q:=2\sqrt{-\log R}$.
Поймём, что
\[
    R\in\left[0;1\right] \Rightarrow \log R \le 0 \Rightarrow 2\sqrt{-\log R} \ge 0
\]
Теперь
\[
\begin{aligned}
\forall q\ge 0 \quad
P\left(Q\le q\right)&=
P\left(2\sqrt{-\log R} \le q\right)=
P\left(-4\log R \le q^2\right)=
P\left(\log R \ge \frac{-q^2}{4}\right)=
\\&=
P\left(R \ge \exp{\frac{-q^2}{4}}\right)=
1-\left(\exp{\frac{-q^2}{4}}\right)^2=
1-\exp{\frac{-q^2}{2}}
\end{aligned}
\]
Значит
\[
 F_Q(q)=
 \begin{cases}
  1-\exp{\frac{-q^2}{2}} &q\ge 0 \\
  0 &q<0
 \end{cases}
\]
а плотность
\[
f_Q(q)=F_Q'(q)=
 \begin{cases}
  q\exp{\frac{-q^2}{2}} &q\ge 0 \\
  0 &q<0
 \end{cases}
\]

А сейчас немного тригонометрии.
\begin{gather*}
 A:=\cos\phi \\
 B:=\sin\phi \\
 \begin{aligned}
 \forall a \in [-1;1] \quad
 P\left(A \le a\right)&=
 P\left(\cos\phi\le a\right)
 \\&=
 P\left(\arccos a \le \phi \le 2\pi - \arccos a \right)
 \\&=
  \frac{(2\pi - \arccos a) - \arccos a}{2\pi}
  \\&=
  1-\frac{\arccos a}{\pi}
 \end{aligned}\\
  F_A(a)=
  \begin{cases}
    0 & a<-1 \\
    1-\frac{\arccos a}{\pi} & a\in\left[0;1\right] \\
    1 & a>1
  \end{cases} \\
  f_A(a)=F_A'(a)=
  \begin{cases}
    \frac{1}{\pi\sqrt{1-a^2}} & a\in\left[0;1\right] \\
    0 & a\notin\left[0;1\right]
  \end{cases}\\
 \text{Аналогично, }\forall b \in [-1;1] \quad
 P\left(B \le b\right)=
 \frac{\arcsin b}{\pi} - \frac{1}{2} =
  1-\frac{\arccos b}{\pi}\\
  F_B \equiv F_A\\
  f_B \equiv f_A
\end{gather*}

Так как $X=AQ$, $Y=BQ$, а распределения у $A$ и $B$ одинаковые,
то значит $X$ и $Y$ тоже одинаково распределены.
Найдём плотность вероятности для $X$:
\[
\begin{aligned}
    f_X(x)&=\int_{-\infty}^{+\infty} f_Q(q)f_A(x/q) \frac{1}{|q|} dq \\
    &=\int_{0}^{+\infty} f_Q(q)f_A(x/q) \frac{1}{q} dq
    & \left(\forall q<0\quad f_Q(q)=0\right) \\
    &=\int_{0}^{+\infty} q\exp{\frac{-q^2}{2}}f_A(x/q) \frac{1}{q} dq
    &\text{Подстановка $f_Q$} \\
    &=\int_{0}^{+\infty} \exp{\frac{-q^2}{2}}f_A(x/q) dq
    &q\frac{1}{q}=1\\
    &=\int_{|x|}^{+\infty} \exp{\frac{-q^2}{2}}f_A(x/q) dq
    & \frac{x}{q}\notin[0;1]\Rightarrow f_A(x/q)=0 \\
    &=\int_{|x|}^{+\infty} \frac{\exp{\frac{-q^2}{2}}}{\pi\sqrt{1-\left(\frac{x}{q}\right)^2}} dq
    & \frac{x}{q}\notin[0;1]\Rightarrow f_A(x/q)=0 \\
    &= \left. \frac{-\exp{\frac{-x^2}{2}}\left( 1-\erf\sqrt{\frac{q^2-x^2}{2}} \right)}{\sqrt{2\pi}}\right|_{q=|x|}^{q=+\infty}\\
    &=\frac{\exp \frac{-x^2}{2}}{\sqrt{2\pi}}
\end{aligned}
\]
А это плотность вероятности для нормального распределения с нулевым математическим ожиданием и единичной дисперсии.
То есть $X$ и $Y$ являются стандартными гауссовскими величинами.




\chapter{
1. Для заданной плотности $p$ наследуйтесь от класса \texttt{rv\_continuous}
 и сгенерируйте с помощью реализованного подкласса для вашего распределения
 $n$ случайных чисел из данного распределения.
 Посмотрите, как будет работать алгоритм при росте $n$ (сразу большим $n$ не делайте). \\
 2. Реализуйте генерацию для вашего распределения с помощью обратной к функции распределения
 (при написании функции для $F^{-1}$ вызов специальных функций, например, квантили стандартного нормального закона, разрешается). Проведите тот же эксперимент.
 3. Выберите еще один метод для генерации случайных чисел (можно, например, rejecting sampling,
 ratio of uniforms или другой метод). Опишите его математическое обоснование.
 Воспользуйтесь реализацией или сами реализуйте. Проведите тот же эксперимент.
\[
    p(x)=\frac{4 \log^3 x}{x} \mathbb{1} \quad \left(x\in\left[1;e\right]\right)
\]
}
\end{document}

